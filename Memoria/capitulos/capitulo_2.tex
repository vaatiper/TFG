\chapter{Herramientas matemáticas}\label{ch:segundo-capitulo}
En este capítulo se recogen los resultados trabajados durante el grado que han sido necesarios para la elaboración del trabajo.
\section{Algunas cuestiones básicas de Teoría de la Medida}
\begin{definicion}
Dado un conjunto no vacío $\Omega$, una $\sigma$-álgebra en $\Omega$ es una familia $\mathcal{A}$ de partes de $\Omega$ que contenga a $\Omega$ y sea estable por complementación y por unión numerable, esto es, $\mathcal{A} \subseteq \mathcal{P}(\Omega)$ es una $\sigma$-álgebra en $\Omega$ si verifica las siguientes propiedades:
\begin{enumerate}
	\item $\Omega\in\mathcal{A}$
	\item $E\in\mathcal{A} \rightarrow \Omega \backslash E \in \mathcal{A}$
	\item Si $E_{n}\in\mathcal{A}$ para todo $n\in\mathds{N}$, entonces $\bigcup_{n\in\mathds{N}}E_{n}\in\mathcal{A}$
\end{enumerate}
Un espacio medible es un par $(\Omega,\mathcal{A})$ donde $\Omega$ es un conjunto no vacío y $\mathcal{A}$ es una $\sigma$-álgebra en $\Omega$. Los elementos de $\mathcal{A}$ se suelen llamar conjuntos medibles.
\end{definicion}

\begin{definicion}
Si $(X,\mathcal{T})$ es un espacio topológico, la $\sigma$-álgebra engendrada por la topología $\mathcal{T}$ recibe el nombre de $\sigma$-álgebra de Borel de $X$ y sus elementos son los conjuntos Borel medibles (o borelianos) de $X$.
\end{definicion}

\begin{definicion}
Sean $(\Omega_{1},\mathcal{A}_{1}), (\Omega_{2},\mathcal{A}_{2})$ espacios medibles. Decimos que una función $f:\Omega_{1}\rightarrow\Omega_{2}$ es medible cuando la imagen inversa por $f$ de cualquier conjunto medible en $\Omega_{2}$ es medible en $\Omega_{1}$.
\end{definicion}

\begin{definicion}
Dado un espacio medible $(\Omega,\mathcal{A})$, una medida en él es una función $\mu:\mathcal{A}\rightarrow[0,\infty]$ verificando: 
\begin{itemize}
	\item $\mu(\varnothing)= 0 $
	\item Si ${E_{n} : n\in \mathds{N} }$ es una familia numerable de elementos disjuntos de $\mathcal{A}$ dos a dos, se tiene $$ \mu(\cup_{n\in\mathds{N}}E_{n}) = \sum_{n=1}^{\infty} \mu(E_{n})$$.
\end{itemize}
\end{definicion}

\begin{definicion}
Un espacio de medida es una terna $(\Omega,\mathcal{A},\mu)$ donde $\Omega$ es un conjunto, $\mathcal{A}$ una $\sigma$-álgebra en $\Omega$ y $\mu$ una medida definida en $\mathcal{A}$. 
\end{definicion}

\begin{teorema}
Existe una $\sigma$-álgebra $\mathcal{M}$ en $\mathds{R}^{d}$ y $m$ una medida definida en $\mathcal{M}$ con las siguientes propiedades: 
\begin{itemize}
	\item $\mathcal{M}$ contiene a la $\sigma$-álgebra de Borel en $\mathds{R}^{d}$.
	\item $m$ es invariante por traslaciones.
	\item Complitud: Si $E\in\mathcal{M}$, $m(E)=0$ y $\mathcal{A}\subseteq E$, entonces $\mathcal{A}\in\mathcal{M}$.
	\item El par $(\mathcal{M},m)$ verificando estas propiedades es único.
\end{itemize}

Los elementos de $\mathcal{M}$ se llaman conjuntos de Lebesgue medibles en $\mathds{R}^{d}$ y m es la medida de Lebesgue en $\mathds{R}^{d}$. 
\end{teorema}

\begin{definicion}
Llamamos función simple positiva a toda función $s:\Omega\rightarrow[0,\infty[$ medible cuya imagen sea un subconjunto finito de $\mathds{R}^{+}$. Notamos por $\mathcal{S}$ al conjunto de las funciones simples positivas en $\Omega$. Si $s(\Omega) = { \alpha_{1},...,\alpha_{n} }$ es una enumeración de los valores que toma s y, para cada $k\in{1,...,n}$, notamos $A_{k}=s^{-1}({\alpha_{k}})$, los conjuntos $A_{k}$ son medibles, forman una partición de $\Omega$ y $s=\sum_{k=1}^{n}\alpha_{k}\chi_{\alpha_{k}}$. Esta expresión recibe el nombre de descomposición canónica de la función simple positiva s. 
\end{definicion}

\begin{definicion}

Sea $s\in \mathcal{S}$ y un conjunto medible E, llamamos integral de Lebesgue de s sobre E (con respecto a la medida $\mu$), a

$$\int_{E}s \mu = \sum_{k=1}^{n}\alpha_{k}\mu(E\cap A_{k}) \in [0,\infty[,$$
donde $s=\sum_{k=1}^{n} \alpha_{k}\chi_{A_{k}}$ es la descomposición canónica de s. Dicha descomposición es única, salvo el orden de los sumandos, por lo que la definición anterior es correcta.

\end{definicion}
\section{Resultados auxiliares de Análisis Matemático}

\begin{teorema}[Teorema de Hausdorff]\label{thm:h01}
Toda biyección lienal entre dos espacios normados de dimensión finita es un isomorfismo. 
\end{teorema}

\begin{teorema}[Teorema de derivación bajo el signo integral]\label{thm:h02}
Sea $U$ un subconjunto compacto de $\mathds{R}^{n}\times\mathds{R}^{m}$, y sea $f: U \rightarrow \mathds{R}$ una función continua en todo $U$ cuya derivada parcial $\frac{\partial f}{\partial y}$ existe y es contínua en $U$. Entonces, si $A$ y $B$ son sobconjuntos con volumen de $\mathds{R}^{n}$ y $\mathds{R}^{m}$ respectivamente, tales que $B$ es abierto y $A\times B \subseteq U$, se tiene que la función $F:B\rightarrow \mathds{R}$ definida por 
\begin{equation}
F(y) = \int_{A} f(x,y) dx
\end{equation}
es diferenciable en $B$ y 
\begin{equation}
F'(y) = \int_{A}\frac{\partial f}{\partial y} (x,y) dx
\end{equation}
para todo $y\in B$.
\end{teorema}

\begin{definicion}
Un funcional sublineal en un espacio vectorial $X$ es una función $p:X\rightarrow \mathds{R}$ tal que
\begin{enumerate}
\item $p(x+y)\leq p(x) + p(y)$ para cualesquiera $x,y\in X$.
\item $p(\alpha x) = \alpha p(x)$, $\forall\alpha \geq 0$ y  $\forall x\in X$.
\end{enumerate}
\end{definicion}

\begin{teorema}[Teorema de extensión de Hahn-Banach]\label{thm:h03}
Sea $X$ un espacio vectorial y $p$ un funcional sublineal en $X$. Si $M$ es un subespacio de $X$ y $g$ es un funcional en $M$ verificando 
$$ Re g(m) \leq p(m), \qquad (m\in M)$$
entonces existe un funcional lineal $f$ en $X$ cuya restricción a $M$ coincide con $g$ y que verifica
$$Re f(x) \leq p(x), \qquad (x\in X)$$
\end{teorema}


\begin{teorema}[Teorema de representación de Riesz]\label{thm:h04}
Sea $X$ un espacio de Hausdorff localmente compacto. Sea $\Lambda$ un funcional lineal positivo en el espacio de funciones continuas en $X$ de soporte compacto, que denotaremos $C_{c}(X)$. Entonces existe una $\sigma$-álgebra $M$ en $X$ que contiene a todos los conjuntos de Borel en $X$, y existe una medida positiva $\mu$ en $M$ que satisface:
\begin{enumerate}
	\item Para todo $K\subseteq X$ compacto, $K\in M$ y $\mu (K)<\infty$.
	\item Para todo $E\in M$, $\mu (E) = \inf\{\mu (V) : E \subset V, V \text{ abierto}\}$.

	\item Para todo $E\in M$ tal que $\mu (E) < \infty$, $\mu (E) = \sup\{\mu (K): K\subset E, K \text{ compacto}\}$.
	\item Para toda $f\in C_{c}(X)$, $\Lambda (f) = \int_{X}f\mu$.
\end{enumerate}
\end{teorema}
\begin{teorema}[Aplicación del teorema de extensión de Tietze]\label{thm:h07}
 Sea $X$ un espacio normado. Sea \(f:A\to \mathds{R}\) una función continua de un subconjunto cerrado A de X en \(\mathds{R}\) con la topología estándar. Entonces, existe una extensión continua de f a X, esto es, existe una función \(F:A\to \mathbb{R}\) continua en todo X con \(F(a) = f(a)\) para todo \(a\in A\). Además, se puede escoger \(F\) tal que \(\sup\{|f(a)|:a\in A\}~=~\sup\{|F(x)|:x\in X\} \). Esto es, si \(f\) está acotada entonces F puede escogerse acotada.
\end{teorema}


\begin{lema}\label{lm:p01}
Sea K un compacto en \(\mathds{R}^{n}, f \in C(K)\). Entonces, existe una función continua \(E(f)\in C(R^{n}))\) tal que: 
\begin{enumerate}
	\item \(f(x) = E(f)(x), \forall x \in K\)
	\item \( \sup_{x\in \mathds{R}^{n}}\{\vert E(f)(x)\vert\} \leq \sup_{x\in K} \{\vert f(x)\vert\}\)
	\item Existe una constante c tal que \[ \sup_{\vert x'-x''\vert < \delta}\left\{ \vert E(f)(x') - E(f)(x'')\vert \right\} \leq c \sup_{\vert x'-x'' \vert < \delta} \left\{\vert f(x')-f(x'')\vert\right\} \quad ( x',x''\in K)\]
\end{enumerate}
\end{lema}

\begin{lema}[Teorema de Ascoli-Azela]\label{thm:h05}
V es un conjunto compacto en C(K) si, y solo si:
\begin{enumerate}

	\item V es cerrado en C(K).
	\item Existe una constante M tal que \(\|f(x)\|_{C(K)} \leq M, \forall f \in V\).
	\item V es equicontinuo. 
\end{enumerate}
\end{lema}

\begin{lema}[Lema de Riemann-Lebesgue]
Si $f\in L^{1}(\mathds{R}^{n})$, entonces 
\begin{equation}
\mathcal{F}(f)(\xi) = \int_{\mathds{R}^{n}}f(x)e^{i\xi\cdot x} dx \rightarrow 0
\end{equation}
 cuando  $\vert \xi \vert \ \rightarrow \infty$. Es decir, $\mathcal{F}(f)\in C_{0}(\mathds{R}^{n})$.
\end{lema}

\begin{teorema}[Teorema de la gráfica cerrada en espacios de Banach]\label{thm:h06}
Sean $X$ e $Y$ espacios de Banach. Entonces, toda aplicación lineal $f$ de $X$ en $Y$ tal que el conjunto
\begin{equation}
    G(f) = \left\{ (x,y)\in X\times Y \text{ : } y=f(x) \right\}
\end{equation}
es cerrado en $X\times Y$, es continua. 
\end{teorema}

\begin{teorema}[Teorema de la convergencia dominada de Lebesgue]

Sea $\{ f_n\}$ una sucesión de funciones medibles definidas en $\Omega$ que converge puntualmente a una función $f$ medible.  Si existe una función $g\in L_1 (\Omega)$ tal que 
\begin{equation}
    \vert f_n (x) \vert \leq g(x) \qquad (\forall n\in \mathds{N}, \forall x\in \Omega)
\end{equation}

entonces $f$ es integrable y, en particular, 

\begin{equation}
    \lim_{n\rightarrow \infty} \int_{\Omega}f_n = \int_{\Omega} f.
\end{equation}
\end{teorema}
